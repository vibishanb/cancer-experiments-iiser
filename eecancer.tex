% !TeX spellcheck = en_GB
\documentclass[12pt, letterpaper, onecolumn]{article}
\usepackage[T1]{fontenc}
\usepackage[utf8]{inputenc}
\renewcommand{\familydefault}{\sfdefault}
%\usepackage{helvet}
\usepackage{amsfonts, amsmath}
\usepackage[backend=biber, style=numeric-comp, citestyle=nature, doi=false, isbn=false, url=false]{biblatex}
\addbibresource{eecancer.bib}
\usepackage[margin=0.75in]{geometry}
\usepackage{hyperref}
\hypersetup{
	colorlinks = true, %Colours links instead of ugly boxes
	urlcolor = blue, %Colour for external hyperlinks
	linkcolor = red, %Colour of internal links
	citecolor = red %Colour of citations
}
\usepackage{graphicx}
\usepackage{fancyhdr}
\pagestyle{fancy}
\fancyhf{}
\lhead{Proposal for experiments}
\rhead{\thepage}
\makeindex
\title{Experimental evolution in mammalian cell lines \\
	\large \textit{Selection for EGF-independent growth in HEK293 cells}}
\author{Vibishan B.}

\begin{document}
	\maketitle
%	\tableofcontents
%	\newpage
	\section{Background}
	\paragraph{\empty}It is generally recognized that cancer progression occurs through the accumulation of mutations in the somatic tissue. Since mutagenesis progresses in a stage-wise manner over time \cite{ARMITAGE1954}, it is possible to envision it as an micro-evolutionary process, with selection in favour of mutations that provide a growth advantage \cite{Nowell1976a}. A wide range of concepts from evolutionary biology and ecology have therefore been exapted to explain various aspects of carcinogenesis and progression \cite{Aktipis2013,Gerlinger2014,Kareva2011,Merlo2006,Davis2017a}, including interactions between tumour and stromal cells, the plausible existence of cancer stem cells, intra-tumour heterogeneity (ITH) and the emergence of drug resistance. A central idea in the application of evolutionary principles in cancer is that of clonal expansion; essentially, oncogenic mutations are generally those that allow the mutant cell to grow at a higher rate than its non-mutant neighbours. This growth difference places the mutant at a significant advantage, and it grows to produce a population of clones, all of which carry the advantageous mutation. It is possible that multiple such mutations independently occur in the tissue, leading to several such clonal populations that compete with each other for resources like space and nutrition. The progression of cancer within an individual is therefore a process of continuous competition among cell populations in response to a wide range of selective forces, in which populations with advantageous mutations grow at the expense of those with lower relative fitness. Under this framework, it can be seen that the canonical ``driver'' mutations that are thought to be necessary for cancer to progress are those mutations that together allow the carrier population to outcompete all other cells in its vicinity. 
	
	\paragraph{\empty}Cellular interactions with the tissue micro-environment are known to affect the behaviour of normal and cancer cells. Breast cancer literature is particularly rich in this regard; levels of estrogen and progesterone affect production of growth factors and ECM components by cells \cite{Haslam2001,Woodward2000,DICKSON1987}, and important observations have also been made suggesting synergistic action between hormonal and cytokine-based regulation of cell growth \cite{Freund2003}. The substantial scope of these interactions has been comprehensively reviewed, both for breast cancer particularly \cite{Hansen2000}, and more generally across cancer types \cite{Pietras2010,Hanahan2012,Cabarcas2011a}. Such interactions leading to a certain contingency of cellular phenotypes, which we will refer to as context dependence, has been observed in a limited sense over time. For instance, estradiol has been shown to promote invasion and metastasis in ER-positive tumours, while inhibiting progression in ER-negative tumours \cite{Garcia1992}. In this case, the prohibitive effect of estradiol in ER-negative tumours points to the possibility that presence of estradiol reduces the selective advantage of an ER-negative phenotype. More recently, experiments have been reported in which behaviourally-enriched environments or physical exercise seemed to show cancer-suppressive effects \cite{Cao2010,Rundqvist2013}. This observation can be explained based on the fact that behaviour and/or physical activity directly affects growth factor secretion through neural and non-neural pathways; as with hormonal regulation, the presence of growth factors in the environment could markedly alter selective outcomes of mutant phenotypes. Another particularly well-recorded case is that of the association between cancer and obesity, which is known to affect cell behaviour and cancer progression through several physico-chemical pathways, both locally and systemically \cite{Druso2018,Iyengar2016}.  
	
	\paragraph{\empty}Together, these observations point towards strong context dependence in cellular behaviour, both in the course of normal physiological functions and cancer progression. This implies that the phenotypic outcome of a mutant is not fixed, but is determined, to a significant extent, by the presence of a conducive micro-environment. Carcinogenesis can then be visualised as an outcome of competition between strongly context-dependent cellular phenotypes, where the ``successful'' progression of a phenotype is determined by the right combination of micro-environmental factors and relatively benign competitors. The qualitative details of this process remain unexplored, and this motivates the current study, which seeks to address the question of context dependence in cancer progression, by exploring how outcomes of simple intercellular competition \textit{in vitro} change relative to the concentration of growth factors in the cell culture medium. 
	
	\section{Rationale}
	
	\paragraph{\empty}Growth factors (GFs) are an important aspect of regulation and restriction of cell proliferation, and it is not surprising that mutations related to GF signalling and GF-independent growth are prevalent in the cancer genome. As discussed earlier, the fitness advantage of a GF-independence mutation is context-dependent. If the medium is deficient in GFs, the mutation brings a selective advantage to the mutant. However, if the medium has sufficient GF to support cell growth, GF independence is no longer advantageous as the mutant faces competition from non-mutant cells that can now grow as well as the mutants. We intend to test this hypothetical picture \textit{in vitro} through experimental competition between two cell populations, one of which can grow independent of a GF while the other depends on the GF for growth. Under the above premise, we expect the mutant population to be successful only under a low GF concentration regime.
	
	\paragraph{\empty}The study design is in two stages: (1) selection for GF independence in a suitable cell line through an experimental evolution approach, and (2) \textit{in vitro} competition between the ``adapted'' and non-adapted cell populations under a range of GF concentration.
	
	
	The HEK293 cell line is well-known for its flexibility and ease of handling, and is used extensively in a wide range of experiments \cite{Thomas2005}. We therefore propose to use this cell line for the first stage of selection.
	
	
	For the first stage, a population of cells is required that can grow without GFs. This would be straightforward if the cell line under study is constitutively dependent on external supplementation of a growth factor like EGF. However, HEK cells do not require such supplementation, and are routinely grown in standard media supplemented with $10\%$ FBS. Removal of FBS should therefore have a similar effect to the removal of a specific growth factor on which the cell line is dependent; this is plausible as FBS itself primarily functions as a source of growth factors or hormones required for growth. While serum starvation is also a frequent procedure in cell culture, its removal is not a simple treatment; because its composition is complex and uncharacterised, FBS cannot be said to play a particular role in cell culture through a particular pathway, which will be affected by its removal. Instead, it is likely that removal of FBS would have several effects, one of which would be to select for growth independent of GFs.
	
	\paragraph{\empty}A reverse approach could be worth considering, in which selection is directed towards EGF dependence rather than independence. In the case of HEK cells, it might be possible to gradually reduce the proportion of FBS in the growth medium, while replacing it with EGF. The optimum proportion of FBS:EGF must be found by trial-and-error so that cell growth remains unaffected. This population of cells would then represent an EGF-dependent state of cell growth, and competition between the selected cells and unselected HEK cells should then show the same kind of context dependence as expected earlier; ultimately, there would still one GF-independent cell population competing with a GF-dependent population under a range of GF concentrations.
	
	\paragraph{\empty}The alternative is to use a cell line like HC11 that is intrinsically dependent on EGF for growth, which can accommodate specific selection for EGF independence \footnote{This section is purposefully indecisive, as more opinions and comments are required to decide which of these three methods should be adopted.}.


%	Independent aim (Aim 3)-compare growth properties of an EGF overexpressing cell line like A431 with that of a normal cell line like WS-1 (same tissue source) in response to EGF concentration.
	
	
%	\section{Planned methodology}
	
%	\subsection{Optimum proportion of EGF and FBS}
%	As mentioned above, HEK cells are not naturally EGF-dependent for their growth. It is therefore necessary to establish a different mode of growth that is dependent on EGF. We propose to do this by reducing the percentage of FBS and attempting to compensate for this through supplementation with EGF.
%	
%	
%	A step-wise procedure:
%	\begin{enumerate}
%		\item[\textbf{Step 1}:] Initial expansion of HEK293 in a t25 flask with DMEM + 10\% FBS, with no EGF
%		\item[\textbf{Step 2}:] Split cells, and culture in a 24-well plate, with four different concentrations of EGF, and six concentrations of FBS each:
%			\begin{itemize}
%				\item EGF: 5, 10, 15, and 20 ng/mL
%				\item FBS: 7.5, 5, 2.5, 1.25, and 0\%
%			\end{itemize}
%		\item[\textbf{Step 3}:] Comparing growth across wells after a sufficient period of time would identify the optimum proportion of EGF and FBS for HEK growth.
%	\end{enumerate}
%
%	\subsection{Establishing control cells}
%	The above design would lead to an optimum proportion of EGF and FBS for HEK growth. The control cells in this case, are the ones not selected for EGF-dependent growth. They must be maintained in that percentage of FBS as identified above; for example, if the first stage identifies 5\% FBS and 5 ng/mL EGF as the optimum proportion, the control cells will be grown in 5\% FBS without any EGF.
%	
%	\subsection{Selection for EGF dependence}
%	\begin{itemize}
%		\item[\textbf{Step 1}:] Seed non-selected HEK cells with the corresponding medium composition as identified in 2.1, in a t25 flask, and grow to confluency.
%		\item[\textbf{Step 2}:] Upon confluency, detach the cells by trypsinisation.
%		\item[\textbf{Step 3}:] Seed the cells at the same density as before in a different t25 flask, with $1.5$ times the concentration of EGF from the previous step, with all other components remaining the same.
%		
%		We continue to use t25 flasks because we need sufficient cell number for effective selection and adaptive response, as mentioned above.
%		\item[\textbf{Step 4}:] Once confluency has been reached in the above culture condition, repeat Steps 2 and 3, but with $2$ times the EGF concentration used in \textbf{Step 1}.
%	\end{itemize}
%	The rationale is to finally create a cell population that requires about twice the starting amount of EGF in the medium for optimum growth. In this case, it may not be necessary to push selection up to twice the starting amount of EGF as it will only enhance the existing difference.
	
	\printbibliography
	
	
\end{document}