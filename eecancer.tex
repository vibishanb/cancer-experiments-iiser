% !TeX spellcheck = en_GB
\documentclass[12pt, letterpaper, onecolumn]{article}
\usepackage[T1]{fontenc}
\usepackage[utf8]{inputenc}
\renewcommand{\familydefault}{\sfdefault}
%\usepackage{helvet}
\usepackage{amsfonts, amsmath}
\usepackage[backend=biber, style=numeric-comp, citestyle=nature, doi=false, isbn=false, url=false]{biblatex}
\addbibresource{eecancer_v4.bib}
\usepackage[margin=0.75in]{geometry}
\usepackage{hyperref}
\hypersetup{
	colorlinks = true, %Colours links instead of ugly boxes
	urlcolor = blue, %Colour for external hyperlinks
	linkcolor = red, %Colour of internal links
	citecolor = red %Colour of citations
}
\usepackage{graphicx}
\usepackage{fancyhdr}
\pagestyle{fancy}
\fancyhf{}
\lhead{Proposal for experiments}
\rhead{\thepage}
\makeindex
\title{Experimental evolution in mammalian cell lines \\
	\large \textit{Genetic vs non-genetic heterogeneity in intercellular competition}}
%\author{Vibishan B.}

\begin{document}
	\maketitle
%	\tableofcontents
%	\newpage
	\section{Background}
	\paragraph{\empty}Contemporary literature in cancer biology is enriched in the somatic evolution view of carcinogenesis and progression. Since mutagenesis progresses in a stage-wise manner over time \cite{ARMITAGE1954}, it is possible to envision it as an micro-evolutionary process, with selection in favour of mutations that provide a growth advantage \cite{Nowell1976a}. A wide range of concepts from evolutionary biology and ecology have therefore been exapted to explain various aspects of carcinogenesis and progression \cite{Aktipis2013,Gerlinger2014,Kareva2011,Merlo2006,Davis2017a}, including interactions between tumour and stromal cells, the plausible existence of cancer stem cells, intra-tumour heterogeneity (ITH) and the emergence of drug resistance. A central idea in this application of evolutionary principles in cancer is that of clonal expansion; essentially, oncogenic mutations are generally those that allow the mutant cell to grow at a higher rate than its non-mutant neighbours. This growth difference places the mutant at a significant advantage, and it grows to produce a population of clones, all of which carry the advantageous mutation. It is possible that multiple such mutations independently occur in the tissue, leading to several such clonal populations that compete with each other for resources like space and nutrition. The progression of cancer within an individual is therefore a process of continuous competition among cell populations in response to a wide range of selective forces, in which populations with advantageous mutations grow at the expense of those with lower relative fitness. Under this framework, it can be seen how canonical ``driver'' mutations can be thought necessary for cancer to progress as those mutations that together allow the carrier population to outcompete all other cells in its vicinity. 
	
	\paragraph{\empty}This largely genetic-adaptationist paradigm, at the tissue level, offers scope for expansion in at least two directions: (1) context dependence of outcomes of cellular interactions, and (2) non-genetic modes of adaptive response.
	
	
	Cellular interactions with the tissue micro-environment are known to affect the behaviour of normal and cancer cells. Breast cancer literature is particularly rich in this regard; levels of estrogen and progesterone affect production of growth factors and ECM components by cells \cite{Haslam2001,Woodward2000,DICKSON1987}, and important observations have also been made suggesting synergistic action between hormonal and cytokine-based regulation of cell growth \cite{Freund2003}. The substantial scope of these interactions has been comprehensively reviewed, both for breast cancer particularly \cite{Hansen2000}, and more generally across cancer types \cite{Pietras2010,Hanahan2012,Cabarcas2011a}. Such interactions leading to a certain contingency of cellular phenotypes, which we will refer to as context dependence, has been observed in a limited sense over time. For instance, estradiol has been shown to promote invasion and metastasis in ER-positive tumours, while inhibiting progression in ER-negative tumours \cite{Garcia1992}. In this case, the prohibitive effect of estradiol in ER-negative tumours points to the possibility that presence of estradiol reduces the selective advantage of an ER-negative phenotype. More recently, experiments have been reported in which behaviourally-enriched environments or physical exercise seemed to show cancer-suppressive effects \cite{Cao2010,Rundqvist2013}. This observation can be explained based on the fact that behaviour and/or physical activity directly affects growth factor secretion through neural and non-neural pathways; as with hormonal regulation, the presence of growth factors in the environment could markedly alter selective outcomes of mutant phenotypes. Another particularly well-recorded case is that of the association between cancer and obesity, which is known to affect cell behaviour and cancer progression through several physico-chemical pathways, both locally and systemically \cite{Yilmaz2012,Beyaz2016,Druso2018,Iyengar2016}. Together, these observations point towards strong context dependence in cellular behaviour, both in the course of normal physiological functions and cancer progression. This implies that the phenotypic outcome of a mutant is not fixed, but is determined, to a significant extent, by the presence of a conducive micro-environment. Carcinogenesis can then be visualised at least partly as an outcome of competition between strongly context-dependent cellular phenotypes, where the ``successful'' progression of a phenotype is determined by the right combination of micro-environmental factors and relatively benign competitors.
%	The qualitative and quantitative details of this process remain unexplored, and this motivates the current study, which seeks to address the question of context dependence in cancer progression, by exploring how outcomes of simple intercellular competition \textit{in vitro} change relative to the concentration of growth factors in the cell culture medium. 
%	

	
	\paragraph{\empty}Plasticity of cellular phenotypes has been a well-recorded phenomenon; differentiated cells de-differentiate to regenerate lost stem cells, or trans-differentiate into other cell lineages, among many other possible trajectories in stem and non-stem cells, even in the adult body \cite{Varga,Tetteh2015}. The balance between symmetric and asymmetric division in stem cells is known to be pliable, with stem cells able to compensate for loss of stem as well as non-stem cells \cite{Morrison2006}, while intestinal crypt cells might share a common ``permissible'' chromatin state that leads to different lineage outcomes under the action of lineage-specific transcription factors \cite{Kim2014a}. Although cell plasticity is most widely discussed in describing the turnover of the intestinal crypt, similar responses have been observed in both in the pancreas, as plasticity among the acinar and ductal phenotypes of exocrine pancreas, as well as regeneration of islet cells from the other pancreatic cells, and the liver, as transitions between hepatocytes and biliary endothelial cells \cite{Yanger2013,Kopp2016}.
	
	Such drastic changes in cellular phenotype point towards a remarkable capability of cells to respond to environmental signals without requiring a specific genetic change. In the context of cancer, the epithelial-mesenchymal transitions are the most widely studied instance of plasticity \cite{Ye2015}. Somatic evolution through non-genetic changes has been shown to contribute significantly to the emergence of drug resistance, at least in leukemic \cite{Pisco2013} and melanoma cells \cite{Su2017}. Indeed, current perspectives are moving towards considering non-genetic drivers of Darwinian selection \cite{Brock2009a}, or even the effects of Lamarckian inductive process alongside conventional mutation-based adaptation \cite{Huang2012a}. The sources of non-genetic heterogeneity are multiple, and it is plausible that these sources are all active during development and somatic evolution. As with the intestinal stem cells, epigenetic changes can broadly enable plastic responses during the course of adaptive change \cite{Flavahan2017}. Cell signaling networks can equally drive phenotypic responses without mutational change due to the potential existence of intermediate metastable stable states in the network that may all differ subtly in their phenotypic outcomes; a possibility that has seen much application of dynamical theory \cite{Jolly2018}.
	
	\newpage
	\section{Rationale}

	Given this background, the question of the balance between genetic and non-genetic paths of adaptive change looms large in somatic evolution and carcinogenesis. We seek to address this question through an experimental evolution framework, in which growth factor (GF) concentrations will be used as the selection force on an \textit{in vitro} population of mammalian cells that are normally GF-dependent.
	
	\paragraph{\empty}Growth factors (GFs) are an important aspect of regulation and restriction of cell proliferation, and as discussed earlier, the fitness advantage of a GF independence mutation is context-dependent. If the medium is deficient in GFs, the mutation brings a selective advantage to the mutant. However, if the medium has sufficient GF to support cell growth, GF independence is no longer advantageous as the mutant faces competition from non-mutant cells that can now grow as well as the mutants. At the same time, it is not necessary for adaptation to occur only by mutation, and the pattern of competitive exclusion suggested above is specific to irreversible genetic change. If the change in cell behaviour is instead reversible, regardless of the exact mechanism, competitive exclusion of non-selected cells by selected ones would be less discrete, as the selected cells could reversibly modify depending on the available GF concentration. The profile of competitive exclusion can therefore be used to distinguish between reversible and irreversible changes in cellular behaviour.
	
	
	\textit{A priori}, it is possible to speculate that the strength of selective force, or the duration for which it is applied can produce cellular responses that are widely distributed in the reversible-irreversible spectrum. For instance, it might be possible that GF deprivation for longer than 3 generations, or a reduction of more than 50\%, would lead to selection of irreversible responses. The strength and duration of selection are then two forms of treatment that can be used in this study.
	
	
	We intend to test this hypothetical picture experimentally in three stages: (1) testing of cell response to a single step of GF reduction by varying proportions, and subsequent selection for growth comparable to control cells, (2) selection under reduction of GF by an intermediate proportion, for different durations, and (3) cell competition by co-culture of selected and control cells.
%	Under the above premise, we expect the mutant population to be successful only under a low GF concentration regime.
	
	\paragraph{\empty}The HEK293 cell line is well-known for its flexibility and ease of handling, and is used extensively in a wide range of experiments \cite{Thomas2005}. We therefore propose to use this cell line for the first stage of selection.
	
	
	Specifically for the selection force, it is important to note that HEK cells do not require external supplementation of any specific GF, but are routinely grown in standard media supplemented with 10\% FBS. Removal of FBS would therefore have a similar effect to deprivation of a GF on which the cell line is dependent; this is plausible as FBS itself primarily functions as a source of growth factors or hormones required for growth. While serum starvation is also a frequent procedure in cell culture, its removal is not a simple treatment; because its composition is complex and uncharacterised, the effects of FBS removal cannot be attributred to a specific component. It should be possible however, to draw general inferences regarding the role played by a conducive medium that provides growth signals. In this regard, there is precedent for such an approach; as early as 2014, measurements of growth rates, separately and in co-culture, have been made for normal cell lines and engineered cell lines dependent on external IGF \cite{Archetti2015}. In this case, cells grown in lower concentrations of FBS showed slower growth as expected, and this retardation was at least partly relieved in IGF-dependent cells upon external IGF supplementation.
	
	
	Both cell viability and doubling time must be measured as these represent two possible modes of adaptive response, as the logistic growth rate or carrying capacity, or together as the growth curve. Since adaptive responses are studied relative to the control cells, direct comparisons can be made by assaying selected and control cells whenever the latter reach 100\% confluency.
	
%	\paragraph{\empty}A reverse approach could be worth considering, in which selection is directed towards EGF dependence rather than independence. In the case of HEK cells, it might be possible to gradually reduce the proportion of FBS in the growth medium, while replacing it with EGF. The optimum proportion of FBS:EGF must be found by trial-and-error so that cell growth remains unaffected. This population of cells would then represent an EGF-dependent state of cell growth, and competition between the selected cells and unselected HEK cells should then show the same kind of context dependence as expected earlier; ultimately, there would still one GF-independent cell population competing with a GF-dependent population under a range of GF concentrations.
	
%	\paragraph{\empty} An earlier attempt provides the starting point for the second approach; experiments had been initiated (results unpublished) to study differences in responses to change in EGF concentration in the growth medium of three different breast-derived cell lines (Figure \ref{fig1}). MCF-10A is a breast epithelial cell line that is known to be dependent on EGF, amongst other factors including FBS for growth \textit{in vitro}, while the breast neoplastic cell lines, MDA-MD-453 and MDA-MB-231, are both capable of EGF-independent growth. The short study showed that the cell lines all had different optimal EGF concentrations at which cell growth peaked. This suggests important qualitative and/or quantitative differences in the mechanisms of EGF dependence between these cell lines. While standard literature is rich in reports of putative mechanisms like autocrine secretion or constitutive activation/expression of various signalling pathway components, their correlations to quantitative differences in GF dependence are yet to be demonstrated clearly. This is therefore a potentially promising line of investigation, where an initial aim could be to examine if the mechanism of GF independence in a given cell line could be used to predict its GF optimum, or general responses to changes in GF concentration.
	
%	\begin{figure}[p]
%		\centering
%		\includegraphics[width=0.75\textwidth]{fig1.png}
%		\caption{\label{fig1} Different growth optima of EGF concentration for three breast-derived cell lines.}	
%	\end{figure}

%	\section{Possible outcomes}
%	A central consideration in predicting outcomes of the above experiments is the cost of GF-independent growth, which in turn determines how and when this cost is offset by the competitive environment. If the cost of GF independence is not met by sufficient advantages, it no longer carries a benefit.
%	
%	This is also related to whether the phenotype of GF-independent growth is reversible or permanent. In the context of cellular signalling, cells continuously and dynamically adjust their metabolism and physiology in response to a wide variety of signals and conditions. However, it is the frequency and magnitude of environmental fluctuations faced by the cell that determines if cellular phenotypes are entirely plastic, or are merely following a pre-specified reaction norm. If cellular response indeed varies along a pre-determined range of phenotypes, or a reaction norm, then external environmental changes within this range are morely likely to cause reversible responses in cellular phenotype. Importantly, this kind of behaviour does not lend into clonal expansion within the tissue milieu; progeny of a cell in a certain state of the reaction norm inherit the reaction norm, and not the exact state of the parent cell. Extending this premise, variance within this reaction norm is not likely to be the basis of intercellular competition, as long as the reaction norm itself is conserved among the competing cells. This case also precludes the notion of a permanent cost for reversible GF independence; if the independence is a plastic response already built into the signalling machinery, it could potentially revert to a GF-dependent state if and when GFs become available.
%	
%	This kind of response along a reaction norm is a strong possibility in the experiment evolution setup; since the variation in FBS is planned to be monotonic, if not linear, a reversible shift in cellular growth phenotype might occur for certain ranges of FBS concentration. The reversability of this phenotype would affect the outcomes of the competition stage experiments. Should the response to FBS deprivation be reversible, ``selected'' cell lines would continue to survive in competition with naive cell lines under FBS-rich conditions by utilising the GFs in the medium. On other hand, if GF independence is acquired irreversibly, this switch might not occur, leading the ``selected'' cell lines to incur significant costs relative to naive cell lines. Therefore, it might be possible to make some inferences about the reversibility of cellular phenotypes acquired through experimental evolution stage based on the outcomes of the competition stage. Clearly, it is also possible that this reversibility is a function of how deviant the selection conditions are relative to the range of the pre-determined reaction norm.
%	
%%	\paragraph{\empty} A corollary to the above scenario is the permanent cost associated with irreversible cellular phenotypes, as with the three breast-derived cell lines described in the second approach. GF dependence is a fixed feature of MCF-10A, much as GF independence is a fixed feature of the other two cells. However, the mechanism of independence becomes important here, and different mechanisms could lead to different expectations. For example, GF independence through autocrine GF secretion or constitutive receptor activation might prove costly under GF-rich conditions, while GFR overexpression might not. Given extensive databases on cell lines, it must be possible to get at this information, in terms of genetic differences like mutations or copy number variations, or transcriptomic expression data. This information would then enable better predictions of the outcomes of this approach. These differences  notwithstanding, it is possible, in principle, to delineate a set of conditions under which an irreversible phenotype can be expected to outcompete a neighbour, which is not the case with reversible phenotypes.
%	
%	\paragraph{\empty} Therefore, while more information is required for more specific predictions of some experiments, it is now possible to say that the outcomes of the experiments could be used to make inferences about the reversibility of adaptive responses to changing growth conditions, and about how reversibility is expected to fare in a competitive setting.
%	
%	\paragraph{\empty} It is possible to consider at least two modes of cellular response-higher survival, or higher proliferation. 
%	The alternative is to use a cell line like HC11 that is intrinsically dependent on EGF for growth, which can accommodate specific selection for EGF independence \footnote{This section is purposefully indecisive, as more opinions and comments are required to decide which of these three methods should be adopted.}.


%	Independent aim (Aim 3)-compare growth properties of an EGF overexpressing cell line like A431 with that of a normal cell line like WS-1 (same tissue source) in response to EGF concentration.
	
	\section{Experimental design}
	
	\subsection{Sampling and assaying for growth rate}
	We need to test for responses in terms of the mean and variance of the distribution of growth rates in the population, and this distribution cannot be constructed with triplicate measurements. Better sampling can be obtained on a 48- or 96-well plate, and a minimum of 15-20 samples per condition would be a reasonable expectation for this distribution. Growth kinetics measurements for these samples can be done using the MTT assay which gives an absorbance read-out, which therefore allows for measurement of cell number over several timepoints.
	
	
	The following work flow could be considered:
	\begin{itemize}
		\item Selection of the cells in 60mm dishes, or t25 flasks as per availability, in the respective conditions.
		\item At the endpoint, use the control cell seeding density to determine a common dilution of the cell suspension for 15-20 samples. Use this dilution for T1-4, and seed the samples in subsequent wells. If 10 points are required for the growth curve, then 10 replicates of each sampling would be required, leading to 150-200 samples for each condition.
		\begin{itemize}
			\item For ease of handling, this can be managed on one 96-well plate per condition, with the wells in each row being samples from the same condition at a given time, and each row being one timepoint for a given condition. This leads to 12 samples and 8 timepoints per condition, and five different plates, one each for C and T1-4.
		\end{itemize}
		\item MTT assays at each timepoint for each sample-plot viable cell number against time for each treatment, and obtain growth rate values from each growth curve.
		\item This leads to a distribution of growth rates for each treatment and the control-compare mean and variance.
	\end{itemize}

	\subsection{Strength of selection}
	\begin{itemize}
		\item All treatments for a single passage
		\item The endpoint is determined by control cells reaching confluency, as unselected cells grown in native levels of FBS are expected to have the best possible growth kinetics, and responses in kinetics are measured relative to this level of growth.
		\item Four treatments:
		\begin{itemize}
			\item T1-1.25\% FBS
			\item T2-2.5\% FBS
			\item T3-5\% FBS
			\item T4-7.5\% FBS
		\end{itemize}
		\item Control (C)-10\% FBS
		\item Seed C and all Ts at the same density, and grow under standard conditions.
		\item Determine endpoint by visual inspection of C for 100\% confluency.
		\item Assay for growth kinetics as in 3.1.
	\end{itemize}

	\subsection{Duration of selection}
	\begin{itemize}
		\item \% FBS for all treatments will be based on the observed response for the strength of selection experiments-intermediate \% FBS is expected to work.
		\item Confluency of control cells is does not mark the endpoint, which is determined by the different treatments, as below.
		\item Four treatments:
		\begin{itemize}
			\item T1-Two passages
			\item T2-Three passages
			\item T3-Four passages
			\item T4-Five passages
		\end{itemize}
		\item Four controls, C1, C2, C3, C4, grown in 10\% FBS for corresponding number of passages as treatments
		\item Assay for growth kinetics as in 3.1, but at each passage for all C1-4 and T1-4; this gives more information about how the growth rate distributions change over time, with the duration of selection.
	\end{itemize}

	\newpage
%	\subsection{Optimum proportion of EGF and FBS}
%	As mentioned above, HEK cells are not naturally EGF-dependent for their growth. It is therefore necessary to establish a different mode of growth that is dependent on EGF. We propose to do this by reducing the percentage of FBS and attempting to compensate for this through supplementation with EGF.
%	
%	
%	A step-wise procedure:
%	\begin{enumerate}
%		\item[\textbf{Step 1}:] Initial expansion of HEK293 in a t25 flask with DMEM + 10\% FBS, with no EGF
%		\item[\textbf{Step 2}:] Split cells, and culture in a 24-well plate, with four different concentrations of EGF, and six concentrations of FBS each:
%			\begin{itemize}
%				\item EGF: 5, 10, 15, and 20 ng/mL
%				\item FBS: 7.5, 5, 2.5, 1.25, and 0\%
%			\end{itemize}
%		\item[\textbf{Step 3}:] Comparing growth across wells after a sufficient period of time would identify the optimum proportion of EGF and FBS for HEK growth.
%	\end{enumerate}
%
%	\subsection{Establishing control cells}
%	The above design would lead to an optimum proportion of EGF and FBS for HEK growth. The control cells in this case, are the ones not selected for EGF-dependent growth. They must be maintained in that percentage of FBS as identified above; for example, if the first stage identifies 5\% FBS and 5 ng/mL EGF as the optimum proportion, the control cells will be grown in 5\% FBS without any EGF.
%	
%	\subsection{Selection for EGF dependence}
%	\begin{itemize}
%		\item[\textbf{Step 1}:] Seed non-selected HEK cells with the corresponding medium composition as identified in 2.1, in a t25 flask, and grow to confluency.
%		\item[\textbf{Step 2}:] Upon confluency, detach the cells by trypsinisation.
%		\item[\textbf{Step 3}:] Seed the cells at the same density as before in a different t25 flask, with $1.5$ times the concentration of EGF from the previous step, with all other components remaining the same.
%		
%		We continue to use t25 flasks because we need sufficient cell number for effective selection and adaptive response, as mentioned above.
%		\item[\textbf{Step 4}:] Once confluency has been reached in the above culture condition, repeat Steps 2 and 3, but with $2$ times the EGF concentration used in \textbf{Step 1}.
%	\end{itemize}
%	The rationale is to finally create a cell population that requires about twice the starting amount of EGF in the medium for optimum growth. In this case, it may not be necessary to push selection up to twice the starting amount of EGF as it will only enhance the existing difference.
	
	\printbibliography
	
	
\end{document}
